\documentclass{article}

% Language setting
% Replace `english' with e.g. `spanish' to change the document language
\usepackage[german]{babel}

% Set page size anad margins
% Replace `letterpaper' with`a4paper' for UK/EU standard size
\usepackage[letterpaper,top=2cm,bottom=2cm,left=3cm,right=3cm,marginparwidth=1.75cm]{geometry}

% Useful packages
\usepackage{amsmath}
\usepackage{csquotes}
\MakeOuterQuote{"}
\usepackage{graphicx}
\usepackage[colorlinks=true, allcolors=blue]{hyperref}

\setlength{\parindent}{0em}

\title{Ethical Evaluation of CoEnv}
\author{Max Richter, and ...}

\begin{document}
\maketitle

\begin{abstract}
Das Ziel dieses Textes ist es, unser Produkt CoEnv unter ethischen Gesichtspunkten zu bewerten. CoEnv ist ein modulares Ökosystem aus kleineren Produkten welche das Zusammenarbeiten in geteilten Umgebungen erleichtern sollen.
\end{abstract}

\newpage

\section{Einleitung}

CoEnv ist ein im Rahmen des Kurses "Functional Objects for Shared Environments" entstandenes Konzept.
\\[2ex]
"Shared Enviroments" sind in unserer Definition Umgebungen oder Orte die Infrastruktur bieten welche Menschen bei der Arbeit unterstützt, meistens Strom, Internet, Küche, Toiletten, Arbeitsplätze und eventuell auch Werkstätten. Dabei gibt es in diesen Umgebungen meistens keine festen Hierarchien.
\\[2ex]
Einige Beispiele für Shared Enviroments wären HackerSpaces, Coworking Spaces und der Open Space des Studienganges Code\&Context an der TH-Köln.
\\[2ex]
Die Vorteile von Shared Enviroments sind vielfältig. Sie ermöglichen ihren Nutzer*innen den direkten Austausch mit anderen Personen aus ihrem professionellen Umfeld. Die gemeinsame Nutzung von Infrastuktur senkt die Ausgaben. Außerdem gibt sie Menschen die zum Beispiel als Freelancer arbeiten eine Möglichkeit zur sozialen Interaktion die manche in ihrem normalen Arbeitsalltag missen.
\\[2ex]
Aus unserer eigenen Erfahrung führen geteilte Umgebungen aber auch oft zu ambiguen Situation in deren die gewünschte oder erlaubte Nutzung von Räumen und Objekten nicht geklärt ist und Unsicherheit entsteht.
\\[2ex]
In eher klassischen Büros ist diese erlaubte Nutzung von Objekten und Räumen relativ klar durch Hierarchien und Zuständigkeitsbereiche geregelt, wie z.B. "das hier ist mein Büro", "der Drucker gehört dem Büro der Grafikabteilung" oder "Dieser Konferenzraum ist jeden Tag von 10-11 Uhr für ein StandUp Meeting belegt".
\\[2ex]
Unser Konzept setzt an diesem Punkt an indem wir versuchen eine Symbolsprache zu entwerfen die ... 
% Symbolsprache wir Kontextabhängig entwickelt... 
% Symbole kommuniziere Eigenschaft/Attribute von Objekten und Räumen
% RoomBoard und Cube kommunizieren dann diese Eigenschaften der Symbole

\newpage

\subsubsection*{RoomBoard} 

Das RoomBoard ist eine interaktive Schaltfläche...

\subsubsection*{Cube}

Der Cube ist eine kleine low-power Arbeitslampe ...

\subsubsection*{Siegel}

Die Siegel sind kleine Symbole...

\subsection{Innovation von CoEnv}

Der interessanteste Aspekt von CoEnv ist der Perspektivenwechsel weg von einer Aufgaben und Task basierten Raumplanung hin zu einer Bedürfnissorientierten.

% Activating Environments, scheinbare Sicherheit durch gelebte Konventionen, Unsicherheit führt zu aktiver Auseinandersetzung mit den Bedürnissen der Mitmenschen.

\subsection{Chancen die durch CoEnv entstehen}

Tbd.

\section{Akteur Analyse}

\subsection{Welche Akteure gibt es?}

\subsubsection{Leitende Personen}
\subsubsection{Gäste}
\subsubsection{Mitarbeitende}
\subsubsection{CoEnv Team}

\subsection{Welche Ziele haben die Akteure?}

tbd.

\subsection{Mit welchen Mitteln sollen diese Ziele erreicht werden?}

tbd.

\section{Folgenanalyse}

\subsection{Positive Folgen die durch CoEnv entstehen}

tbd.

\section{Zusatzarbeit...}
% \cite{greenwade93}

% \bibliographystyle{alpha}
% \bibliography{sample}

\end{document} 
